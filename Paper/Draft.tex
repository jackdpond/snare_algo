%%
%%                  TEMPLATE for Math 402/436 project report
%%
\documentclass[11pt]{amsart}
%%% WARNING: Do NOT change the page size, fonts, or margins!  Penalties will apply.


\usepackage{graphicx}
\usepackage{amssymb,amsmath,amsthm}
\usepackage{placeins} %enables \FloatBarrier, which prevents figures and tables from going below it.
\usepackage{hyperref} %makes cross references into hyperlinks. 
\usepackage{caption}
\usepackage{subcaption} %Allows subfigures

\usepackage{algorithm}
\usepackage{algpseudocode} 
 

%%% WARNING: Do NOT change the page size, fonts, or margins!  Penalties will apply.
%%% WARNING: Do NOT change the page size, fonts, or margins!  Penalties will apply.

\begin{document}

\title{Wildcard Weighting}
\author{Place Names of Group Members Here}

% Change the date to match the date you actually wrote this paper
\date{4 October 2024} % or use \today

\maketitle % this actually makes the title

\begin{abstract}
  The MLB postseason is known to be unpredictable. Over the last decade, Wild Card teams have outperformed division 
  winners. This paper aims to investigate and quantify the underlying factors driving this phenomenon. 
  Using a dataset of MLB games from 2015–2025, we train both Random Forest and XGBoost models to identify the variables 
  that most strongly influence postseason success and to evaluate whether Wild Card teams possess structural advantages 
  that are not captured by regular season standings.

\end{abstract}

%% First Section
\section{Research question and overview of the data }
Recent MLB postseasons have shown an interesting trend: Wild Card teams often
perform better than division winners, even though they typically finish the
regular season with weaker records. This raises an important question for both
baseball fans and analysts:

\begin{quote}
  \textit{What makes a Wild Card team outperform a division winner in the MLB
  postseason?}
  \end{quote}

This question is interesting because it challenges the assumption that
regular season strength is a reliable predictor of postseason performance.
It also relates to the design of playoff formats and whether certain types of
teams are structurally advantaged in short playoff series.

\section{Background and Prior Work}

Several previous studies have examined MLB postseason performance. Baseball
Prospectus and FanGraphs have written extensively about the role of randomness,
bullpen strength, and matchups in postseason series. Academic work (e.g., Berri \&
Schmidt, 2010) has shown that regular season winning percentage is a relatively
weak predictor of postseason outcomes. 
%% Second Section 
\section{Data Set}

We use MLB game data from 2015-2025. The data includes variables such as team OOA, team WOBA, bullpen FIP, bullpen usage,
whether the team is a wildcard, and postseason results. The data set is strong in that the data is specifically geared 
towards matchups and understanding the impact of the wildcard on the postseason. However, with only so many postseason games (317),
it is difficult to fully understand the impact of the wildcard teams.

Based on our prior experience with baseball, we form the following hypotheses:

\begin{enumerate}
    \item Wild Card teams may enter the later rounds of the postseason carrying more momentum than their division winner counterpart, causing them to win.
    \item Bullpen strength will be more important in the postseason than in the regular season.
    \item Matchup-based factors (pitcher handedness, lineup construction) may
    be more important than overall regular season strength. 
\end{enumerate}
]


\section{Data Cleaning / Feature Engineering }

  Tell us what you did when you were cleaning your data and engineering features.  Why did you make the choices that you did?  What are the consequences of those choices?

  \begin{enumerate}
  \item \textbf{FIP} stands for Fielding Independent Pitching, and is a metric designed to capture how well a pitcher prevents runs independently of the rest of the defense. It is a better metric of a pitcher's skill than Earned Run Average because it disregards events outside of the pitcher's control.
  \item \textbf{Bullpen} Because pitching strains a pitcher's arm quickly, one pitcher will start and throw about 5-7 innings before being switched out for a \textit{relief pitcher}. The bank of pitchers ready to relieve are called the Bullpen.
  \item \textbf{OOA} stands for Outs Above Average and signifies how many more outs a player makes than an average fielder at the same position.
  \item \textbf{WOBA} stands for Weighted On-Base Average and signifies how often players make it on base, weighted by how likely they are to score based on the manner in which they made it on base.
\end{enumerate}


  \par We gathered the data we used in three main pulls. Our final dataset incorporates data from 1) The pitcher logs for each game for each pitcher during the regular season from 2014-2015, 2) the pitcher logs for each game for each pitcher during the post season from 2014-2015, and 3) team level statistics (such as outs above average or weighted on-base average) for each team for each year from 2014 - 2025.

  \par Early on, we decided that we wanted our match-up model to include the following features for both teams: starting pitcher's FIP, starting pitcher's freshness, an aggregated value for the FIP and freshness of the pitchers in the bullpen, average OOA (as an indication of the efficacy of the team's defense), average WOBA (as an indication of the efficacy of the team's offense), and whether the team is a wildcard in the playoffs. We also decided to include a feature indicating which team played at home.

  \par The pitcher's logs (collectively referred to as df) included the following pertinent features: Date, Team, a unique Pitcher ID, FIP, Innings played, and the result of the game. In order to assemble it into a useable format for our problem, we made a few tweaks and added some new columns. We first converted the Date column to pandas.DateTime. After sorting the df by Pitcher Id and Date, we grouped the df by Pitcher ID and shifted the Date column to create a Previous Game Date column, and subtracted the Previous Game Date column from the Date column to create a Freshness column for each pitcher in each game.

  \par Additional changes to make the pitcher logs more useful included adding a Starter feature that one-hot-encodes whether the pitcher started that game, and a Game ID to uniquely identify each game, based on the date and the two teams playing.

  \par The dataset of team level statistics by year included the following pertinent features: Team, Year, OOA (averaged over the season), WOBA (averaged over the season), and Is-Wildcard. This dataset was nearly immediately useable after cleaning the Team names to match those in the pitcher logs dataset. We merged it with the pitcher logs on Year and Team.

  \par Equipped with a newly-merged datset (with features Date, Team, Pitcher ID, FIP, Innings played, Game Result, Freshness, Starter, Game ID, Team OOA, Team WOBA, Is-Wildcard for each pitcher in each game) we were ready to assemble a final games dataset. To do so, we grouped the newly-merged dataset by Game ID and then by team. For each team in each game, we assembled a dictionary containing the starter's FIP and freshness, the meaned FIP and freshness of the relief pitchers, whether the team was at home, team OOA, team FIP, whether it was a post-season game, and if so, whether the team was a wildcard during the game. After shuffling the order of the teams, we concatenated the two team's stats into a single row, with a results feature signifying whether the first team in the row won or lost. There were no ties in our dataset. 

\par With our data in a tabular dataset with a binary label, we were ready to train!



%% Third Section
\section{Data Visualization and Basic Analysis}
Analyze the data, draw conclusions, and effectively communicate your main observations and results.
\begin{itemize}
    \item Calculate appropriate summary statistics.
    \item Use appropriate plotting techniques, visualizations, and other tools and techniques you have learned, to thoughtfully identify and evaluate what the data are telling you,how well suited the data are to answering your problem,

    \item Reference figures and plots, like Figure~\ref{fig:MeanSquaredError}.
\end{itemize}



%%Fourth Section
\section{Learning Algorithms and In-depth Analysis }

 Analyze the data using the machine learning techniques discussed in class. Explain what research questions you can answer using the  machine learning techniques presented this semester, and if applicable, what you think you may be able to answer next semester.

Be able to explain the results of your analysis, whether the results are meaningful, and why you chose the tools that you used.  

%%Fifth Section
\section{Ethical Implications and Conclusions}
Thoughtfully analyze the ethical implications of your research questions, the data you gathered, and the analysis that was performed.  Are there privacy or other implications from the collection or use of the data?  Could your results and methods be misused or misunderstood?  What can and should be done to prevent misuse and misunderstanding?  Could your algorithms and methods result in a destructive self-fulfilling feedback loop?  How could that be prevented or controlled?  What other ethical implications does your work have?  


This part should all be done before you get to \emph{page 5}.  The bibliography can spill on to page 6, but we won't read text that goes past page 5.


%%%%%%%%%%%%%%%%%%%%%%%%%%%%%%%%%%%%%
%% Bibliography below
%%%%%%%%%%%%%%%%%%%%%%%%%%%%%%%%%%%%%
\FloatBarrier % Keep the figures from being put after the bibliography
\newpage
%% If using bibtex, leave this uncommented
\bibliography{refs} %if using bibtex, call your bibtex file refs.bib
\bibliographystyle{alpha}

%% If not using bibtex, comment out the previous two lines and uncomment those below
%\begin{thebibliography}{99}
%\bibitem{Vandermeersch} First reference goes here
%\end{thebibliography}

\end{document}
