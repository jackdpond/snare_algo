%%
%%                  TEMPLATE for Math 402/436 project report
%%
\documentclass[11pt]{amsart}
%%% WARNING: Do NOT change the page size, fonts, or margins!  Penalties will apply.


\usepackage{graphicx}
\usepackage{amssymb,amsmath,amsthm}
\usepackage{placeins} %enables \FloatBarrier, which prevents figures and tables from going below it.
\usepackage{float} %enables [H] placement option for figures
\usepackage{hyperref} %makes cross references into hyperlinks. 
\usepackage{caption}
\usepackage{subcaption} %Allows subfigures

\usepackage{algorithm}
\usepackage{algpseudocode} 

\usepackage{tikz}
\usetikzlibrary{positioning}
\tikzset{round/.style={draw,rounded corners,minimum width=3.8cm,minimum height=0.9cm,align=center}}

 

%%% WARNING: Do NOT change the page size, fonts, or margins!  Penalties will apply.
%%% WARNING: Do NOT change the page size, fonts, or margins!  Penalties will apply.

\begin{document}

\title{Wildcard Weighting}
\author{Jack Cook and Jackson Pond}

% Change the date to match the date you actually wrote this paper
\date{4 October 2024} % or use \today

\maketitle % this actually makes the title

\begin{abstract}
  The MLB postseason is known to be unpredictable. Over the last decade, Wild Card teams have outperformed division winners. This paper aims to investigate and quantify the underlying factors driving this phenomenon. Using a dataset of MLB games from 2014–2025, we train both Random Forest and XGBoost models to identify the variables that most strongly influence postseason success and to evaluate whether Wild Card teams possess structural advantages that are not captured by regular season standings.

\end{abstract}

%% First Section
\section{Research question and overview of the data }

\par In Major League Baseball, the wild card teams are those who are have comparitively weaker records than
the division winners, but are selected to advance to the postseason. Over the last decade, wild card teams have performed surprisingly well in the postseason, often beating division winners. This raises important questions for both baseball fans and analysts:

\begin{quote}
  \textit{How should the fact that a team is a Wild Card factor in to predicting their success?}
  \end{quote}

This question is interesting because it challenges the assumption that regular season success is a reliable predictor of postseason performance. 
It also relates to the design of playoff formats and whether certain types of teams are structurally advantaged in short playoff series.

\section{Background and Prior Work}

Several previous studies have examined what features most strongly influence postseason success. FanGraphs applied 
clustering models to predict what bullpen matchups with the highest probability of postseason wins \cite{fangraphs_ps}. Academic work (e.g., Berri \& Schmidt, 2010) has shown that regular season winning percentage is a 
relatively weak predictor of postseason outcomes \cite{berri_schmidt_2010}. 

%% Second Section 
\section{Data Set}

\par We use MLB game data from 2014-2025. The data includes variables such as pitcher skill level and team OOA, team WOBA, bullpen FIP, bullpen freshness, and whether the team is a wildcard (lingo will be explained later in the paper). The data set is strong in that the data is specifically geared towards matchups and understanding the impact of the wildcard on the postseason.

\par We gathered the data for the project in three main pulls. Our final dataset incorporates data from 1) The pitcher logs for each game for each pitcher during the regular season from 2014-2025, 2) the pitcher logs for each game for each pitcher during the post season from 2014-2025, and 3) team level statistics (such as outs above average or weighted on-base average, whether the team is a wildcard) for each team for each year from 2014 - 2025.

\par A shortcoming of this dataset is that the quantity of postseason games (317) is dwarfed by the quantity of regular season games (~20,000). Because the state of whether a team is a wild card or not is only pertinent in the postseason, this dataset (which reflects the real life proportion) is wildly unbalanced. Therefore any models we train may be incentivized to disregard the wildcard feature.

Based on our prior experience with baseball, we form the following hypotheses:

\begin{enumerate}
    \item Wild Card teams may enter the later rounds of the postseason carrying more momentum than their division winner counterpart, causing them to win.
    \item Bullpen strength will be more important in the postseason than in the regular season.
    \item Matchup-based factors (pitcher handedness, lineup construction) may be more important than overall regular season strength. 
\end{enumerate}

\section{Data Cleaning / Feature Engineering }

\par There is an extensive tradition of data science applied to Baseball. This project has benefited from that tradition, in that there are many well developed statistics collected and reported for the MLB. Part of this extensive tradition is an extensive lingo. We define terms that are important for this paper to begin this section.

  \begin{enumerate}
  \item \textbf{FIP} stands for Fielding Independent Pitching, and is a metric designed to capture how well a pitcher prevents runs independently of the rest of the defense. It is a better metric of a pitcher's skill than Earned Run Average because it disregards events outside of the pitcher's control.
  \item \textbf{Bullpen} Because pitching strains a pitcher's arm quickly, one pitcher will start and throw about 5-7 innings before being switched out for a \textit{relief pitcher}. The bank of pitchers ready to relieve are called the Bullpen.
  \item \textbf{OAA} stands for Outs Above Average and signifies how many more outs a player makes than an average fielder at the same position.
  \item \textbf{WOBA} stands for Weighted On-Base Average and signifies how often players make it on base, weighted by how likely they are to score based on the manner in which they made it on base.
\end{enumerate}

\par Early on, we decided that we wanted our match-up model to include the following features for both teams: starting pitcher's FIP, starting pitcher's freshness, an aggregated value for the FIP and freshness of the pitchers in the bullpen, average OAA (as an indication of the efficacy of the team's defense), average WOBA (as an indication of the efficacy of the team's offense), and whether the team is a wildcard in the playoffs. We also decided to include a feature indicating which team played at home.

\par The pitchers' logs (sourced from Baseball-Reference\cite{baseballref}) included the following pertinent features: Date, Team, a unique Pitcher ID, FIP, Innings played, and the result of the game. In order to assemble it into a useable format for our problem, we made a few tweaks and added some new columns. We first converted the Date column to pandas.DateTime objects. After sorting the pitchers' logs by Pitcher Id and Date, we grouped the dataframe by Pitcher ID and shifted the Date column to create a Previous Game Date column, and subtracted the Previous Game Date column from the Date column to create a Freshness column for each pitcher in each game. A pitchers' first game (originally with freshness NaN) was assigned the maximum value for freshness in the rest of the dataset.

\par Additional changes to make the pitcher logs more useful included adding a Starter feature that one-hot-encodes whether the pitcher started that game, and a Game ID to uniquely identify each game, based on the date and the two teams playing.

\par The dataset of team level statistics by year (sourced from Baseball Savant\cite{baseballsavant}) included the following pertinent features: Team, Year, OOA (averaged over the season), WOBA (averaged over the season), and Is-Wildcard. This data was quite clean, except that it was missing all WOBA values for 2024. We decided the lowest impact filler would be the average of all WOBA values. After readying the data, we merged it with the pitcher logs on Year and Team.

\par Equipped with a newly-merged datset (with features Date, Team, Pitcher ID, FIP, Innings played, Game Result, Freshness, Starter, Game ID, Team OOA, Team WOBA, Is-Wildcard for each pitcher in each game) we were ready to assemble a final games dataset. To do so, we grouped the newly-merged dataset by Game ID and then by team. For each team in each game, we assembled a dictionary containing the starter's FIP and freshness, the meaned FIP and freshness of the relief pitchers, whether the team was at home, team OOA, team FIP, whether it was a post-season game, and if so, whether the team was a wildcard during the game. After shuffling the order of the teams, we concatenated the two team's stats into a single row, with a results feature signifying whether the first team in the row won or lost. There were no ties in our dataset. There was a single row with missing data, which we dropped as it was such a small subset of the total data.

\par This means that in the final dataset, there is a row for each game or match-up. Each row stats for each teams that were true \textit{by the end of the game}. For instance, the FIP value in the game's row is the FIP of the pitcher calculated at the end of the game. If this were a prediction model, this would be considered a mild form of data leakage. Since our model is not a prediction model, but rather a model trained in order to evaluage feature importance, we may treat these post-game calulated values as true values. The same concept applies to why we evaluated the freshness and skill (FIP) of the pitchers that actually played in the game rather than the pitchers in the bullpen that were available.

\par With our data in a tabular dataset with a binary label, we were ready to train!

\FloatBarrier % Prevent figures from floating into the next section

%% Third Section
\section{Data Visualization and Basic Analysis}

\par Our final dataset is large, with 19,781 rows, one per game, and 20 features, namely \textbf{Game ID}, \textbf{1 Starter FIP}, \textbf{1 Starter Freshness}, \textbf{1 Relief FIP}, \textbf{1 Relief Freshness}, \textbf{1 WOBA}, \textbf{1 OOA}, \textbf{1 Home} (Is team 1 at home?), \textbf{1 Team}, \textbf{1 Is Wildcard}, \textbf{2 Starter FIP}, \textbf{2 Starter Freshness}, \textbf{2 Relief FIP}, \textbf{2 Relief Freshness}, \textbf{2 WOBA}, \textbf{2 OOA}, \textbf{2 Team}, \textbf{2 Is Wildcard}, \textbf{Is Playoff Game}, and \textbf{Result} (Did Team 1 win?). In this section, we will visually inspect the data and evaluate our initial assumptions.

\par We will first inspect the distributions of key numerical features Starter FIP, Starter Freshness, Relief FIP, Relief Freshness, WOBA, and OOA. We will only look at team 1 since team 1 is a random half of the data, and the corresponding values and distributions for team 2 match.

\par As seen in Figure ~\ref{fig:FeatureDistributions}, skill metrics (FIP, WOBA, OOA) are normally distributed, which is as expected. The distribution of freshness, both for starting pitchers and relief pitchers, is heavily skewed towards 0 days of rest. 


\begin{figure}[htb]
\begin{center} %Put your images in a figure like this
\includegraphics[width=\textwidth]{figures/feature_distributions.pdf} % Better to make them pdfs than png or gif or jpeg
\end{center}
\caption{Distributions of various of the features of our final dataset. Each of the skill metrics (FIP, WOBA, OOA) are distributed normally, as we would expect, and the freshness is greatly skewed towards 0 days rest.  
}
\label{fig:FeatureDistributions} % for automatic cross referencing
\end{figure}

\par Figure ~\ref{fig:ResultBoxPlots} shows the distributions of each of the six key features grouped by result. As shown in the figure, the distributions for each feature hardly vary by result, implying that Baseball data is quite noisy, with lots of randomness incorporated into the final result.

\begin{figure}[htb]
\begin{center} %Put your images in a figure like this
\includegraphics[width=\textwidth]{figures/results_box_plots.pdf} % Better to make them pdfs than png or gif or jpeg
\end{center}
\caption{This figure shows the distribution of six features, grouped by result (1 for win, 0 for loss). Regardless of feature, the result-grouped distributions are nearly identical for each feature. 
}
\label{fig:ResultBoxPlots} % for automatic cross referencing
\end{figure}
% \par 
% \begin{figure}[htb]
% \begin{center} %Put your images in a figure like this
% \includegraphics[width=\textwidth]{figures/home_away.pdf} % Better to make them pdfs than png or gif or jpeg
% \end{center}
% \caption{Surprisingly, teams are slightly more likely to win while playing away games, and slightly more likely to lose at home games. 
% }
% \label{fig:HomeAway} % for automatic cross referencing
% \end{figure}

\par Figure ~\ref{fig:LogisticCorrelation} also shows the weakness of logistic correlation between the \textit{differences} between the two teams' features and the results of the game. The results are a 1 if team 1 wins and 0 if team 1 loses. Each of the independent variables is the value for team 2 subtracted from the value for team 2. As epected, the team with higher WOBA and higher OAA is expected to win. Mysteriously, a pitcher with a higher starting FIP than the opponent is inversely correlated with games won.
\begin{figure}[htb]
\begin{center} %Put your images in a figure like this
\includegraphics[width=\textwidth]{figures/logistic_correlation.pdf} % Better to make them pdfs than png or gif or jpeg
\end{center}
\caption{These figures plot the result of a game (1 for win, 0 for loss) against the difference between team 1 and team 2 for various features: Starter FIP, WOBA, and OOA. We hypothesized that each feature would correlate with wins, and are surprised that Starter FIP is inversely correlated with wins.
}
\label{fig:LogisticCorrelation} % for automatic cross referencing
\end{figure}

\par Considered as a whole, the data is clean and comprehensive, including nearly every game from 2014-2025. The feature values are either numerical ratio data or boolean expressed as 0 or 1, and the distributions of each feature are as expected. It is ideal for training data.

\FloatBarrier % Prevent figures from floating into the next section
\clearpage % Start the next section on a fresh page

%%Fourth Section
\section{Learning Algorithms and In-depth Analysis }
\par For our project, we trained both Random Forest and XGBoost models to predict the outcome of a game based on the features of the teams playing. The primary goal of our modeling was not prediction accuracy, but rather to identify which features most strongly influence game outcomes in the postseason. These models work well for our problem because they are able to return feature importance scores that are interpretable and easy to understand.

\par With different iterations, our best Random Forest model achieved an F1 score of 0.6963, while our best XGBoost model achieved an F1 score of 0.701. The two models had similar feature importance ranking, with the top 15 features being the same for both models.
We found that the most important features in the postseason were a team's FIP and freshness for both starting and relief pitchers. This was somewhat surprising to us, as we expected that a team's OOA and 
WOBA would be more significant in October. However, our results are consistent with baseball intuition: pitching wins games. We also found that being a Wild Card team was not nearly as significant as we expected, ranking last in feature importance. 
See Figure~\ref{fig:PostseasonPermImportance}.

\begin{figure}[H]
  \begin{center}
    \includegraphics[width=0.9\textwidth]{../top15Features.png}
  \end{center}
  \caption{Feature importance for the postseason (XGBoost model)}
  \label{fig:PostseasonPermImportance}
  \end{figure}


\par Taken together, these results suggest what teams should try to optimize for in the postseason. Having a talented, well rested bullpen and a strong starting pitcher appears to be a key ingredient for postseason success. This is easier said than done, as it requires managing the bullpen in a way that does not wear pitchers out over the course of a 162-game season while still winning enough games to reach October in the first place. There has been significant research
into optimal bullpen management, but elaborating further is outside the scope of this project. 


%%Fifth Section
\section{Ethical Implications and Conclusions}
\par The ethical implications of this project are relatively minimal. The data is publicly available and we did not collect any new data. The models we trained are used to identify feature importance and not used in a way that is harmful or misleading. Therefore, there is no risk of misuse or misunderstanding.

\par In conclusion, our project revealed that postseason pitching is the most important factor in determining success. Our original hypothesis that Wild Card teams would perform better than division winners was not supported by our findings. 


%%%%%%%%%%%%%%%%%%%%%%%%%%%%%%%%%%%%%
%% Bibliography below
%%%%%%%%%%%%%%%%%%%%%%%%%%%%%%%%%%%%%
\FloatBarrier % Keep the figures from being put after the bibliography
\newpage
%% If using bibtex, leave this uncommented
\bibliography{refs} %if using bibtex, call your bibtex file refs.bib
\bibliographystyle{alpha}

%% If not using bibtex, comment out the previous two lines and uncomment those below
%\begin{thebibliography}{99}
%\bibitem{Vandermeersch} First reference goes here
%\end{thebibliography}

\end{document}
